\documentclass[a4paper,11pt]{article}

\usepackage[left= 1.5cm,text={18cm, 25cm},top=2.5cm]{geometry}
\usepackage[utf8]{inputenc}
\usepackage[T1]{fontenc}
\usepackage{times}
\usepackage{paralist}
\usepackage{graphicx}
\usepackage{textcomp}
\usepackage{enumitem}
\usepackage{amssymb}
\usepackage{amsmath}
\usepackage{xcolor}
\usepackage[ddmmyyyy]{datetime}
\usepackage{array}
\pagestyle{plain}
\pagenumbering{arabic}
\usepackage[czech]{babel}
\usepackage{lmodern}
\usepackage{float}


\renewcommand*\contentsname{Obsah}

\newdateformat{mydate}{\twodigit{\THEDAY}.{ }\shortmonthname[\THEMONTH] \THEYEAR}

\pagenumbering{arabic}

\usepackage{url}
\DeclareUrlCommand\url{\def\UrlLeft{<}\def\UrlRight{>} \urlstyle{tt}}

% \usepackage{indentfirst}

\begin{document}
\selectlanguage{czech}

\begin{titlepage}
\begin{center}
    {\Huge \textsc{Vysoké učení technické v Brně}}
\vspace{\stretch{0.01}}
    
    {\LARGE \uppercase{FAKULTA INFORMAČNÍCH TECHNOLOGIÍ}}
    
\begin{figure}[h]
\vspace{5.0cm}
\centering
\includegraphics[scale=0.15]{logo.png}
\vspace{-10.0cm}
\end{figure}
    
\vspace{\stretch{0.382}}
	{\LARGE Projekt IMS, 2019Z}
\vspace{\stretch{0.02}}

	{\Huge \textbf{10 - Celulární automaty}}
\vspace{\stretch{0.02}}\\

{\LARGE {Hrabošová krize}}\\

\begin{figure}[h]
\centering
{\Large {\mydate\today}}
\vspace{6cm}
\end{figure}

\end{center}
\begin{compactitem}
\item[] \textbf{Autoři:}
\item[] Marek Petr, xmarek66
\item[] Vanický Jozef, xvanic09
\end{compactitem}

\end{titlepage}

\tableofcontents
\newpage

\section{Úvod}
Táto práca vznikla v rámci predmetu Modelování a simulace na Fakulte informačných technológií VUT v Brne. Práca popisuje model (\cite{slajdy}, snímok č. 7) celulárneho automatu (\cite{slajdy} č. 209), ktorého úlohou je predikcia intenzity populácie hraboša poľného \textit{(Microtus arvalis)} a jeho regulácia. Cieľom práce je porovnanie výsledkov experimentov - jednotlivých protiopatrení, voči hrabošovi poľnému na modeli popísanom v článku \verb|\cite{}|. Ide o protiopatrenia hĺbkovej orby, plytkej orby a použitie chemickej látky - Stutox II s účinnou látkou Fosfidom zinečnatým. Tieto protiopatrenia sú ďalej popísané v článkoch \verb|\cite{}|\verb|\cite{}|. 


Zmyslom experimentov je demonštrovať účinnosť jednotlivých protiopatrení na populáciu hraboša poľného. Predmetom skúmania je grafický výstup zobrazujúci nárast a pokles hustoty populácie hraboša poľného v jednotlivých mesiacoch na poli o veľkosti 1 hektáru.

\subsection{Zdroje faktů}
Ako zdroje informácií boli použité odborné publikácie zaoberajúce sa problematikou premnoženia hraboša poľného, vedecké články zaoberajúce satvorbou celulárneho automatu zameraného na túto problematiku. Dôveryhodnosť informácií bola overovaná vyhľadávaním tychto informácií v iných odborných publikáciách a potvrdená odborníkmi z praxe. \verb|\cite{}|\verb|\cite{}|\verb|\cite{}|\verb|\cite{}|

\subsection{Ověření validity/funkčnosti}
Overovanie validity modelu bolo vykonávané priebežne simulácia modelu zodpovedala správaniu populácie hraboša poľného ako to popisujú publikácie \verb|\cite{}|\verb|\cite{}|\verb|\cite{}|. Úbytok populácie v zimnom období a populačná explózia prejavujúca sa od jarných mesiacov zodpovedá aj nameraným datam\verb|\cite{}|. Overovanie experimentov bolo vykonané na základe odborných článkov popisujúcich približne rovnakú účinnosť jednotlivých protiopatrení a konzultáciou s odborníkom z Agropriemyslu - Gabriel Koncz z Poľnohospodárskeho družstva so sídlom v Períne na Slovensku.

\section{Rozbor tématu, použitých metod a technologií TODO čísla} 
K vytvoreniu modelu populačnej explózie je potreba vedieť údaje o hrabošovi poľnom \textit{(Microtus arvalis)} a jeho chovaní v priebehu roka. Zároveň je potreba vedieť ako ovplivňujú hraboša poľného zimné obdobia. Rozmnožovacie obdobie je v treťom až desiatom mesiaci. Dĺžka tehotenstva je 19 až 21 dní. Veľkosť vrhu za rok u jednej samice je 1 až 12 mláďat, avšak najčastejšie 5 až 6. Samica vrhne za rok 1 až 4 krát, pričom mláďa pohlavne dospieva vo veku minimálne dvoch týždňov. Priemerná dĺžka života hraboša poľného je 2,5 mesiaca. \verb|\cite{}|. Na jednom hektári vyskytuje maximálne 3000 až 7000 jedincov. \verb|\cite{} TODO| V populácii hrabošov tesne prevyšujú samice a to približne s 60\% prevahou. 


\subsection{Popis použitých postupů}
Pri práci byl využit objektově orientovaný jazyk C++. Tento jazyk je díky své rychlosti vhodný pro vytváření simulací, které mohou být výpočetně náročné a jazyk s nižší rychlostí výpočtu by mohl simulaci výrazně zpomalit.


K vizualizaci jednotlivých stavů celulárního automatu byla použita grafická knihovna OpenCV2 \verb|\cite{}|, ve které lze snadno implementovat vytváření jednoduchých obrázků.

\subsection{Popis použitých metod a technologií}



\section{Koncepce modelu}
todo

\section{Architektura simulačního modelu}
Simulátor je složen ze dvou tříd \emph{Cell},\emph{Grid} a \emph{Image}. Mřížka celulárního automatu má rozměry 100x100 a představuje jeden ha. Počáteční stav automatu je generován náhodně. Simulace vždy běží po dobu 48 měsíců.

\subsection{Třída Cell}
Třída \emph{Cell} implementuje jednu buňku celulárního automatu. Obsahuje informace o stavu této buňky, tedy její souřadnice a hustotu. Buňka odpovídá 1x1m pole.

\subsection{Třída Grid}
Třída grid reprezentuje mřížku celulárního automatu a zodpovídá za jeho chování. Také obsahuje základní informace modelu. Například velikost mřížky, porodnost, úmrtnost a další. Dále se v ní nachází metody umožňující spuštění a řízení simulace. V této třídě se také nachází metoda emph{init\_present\_grid()}. Ta vytváří čtyřicet shluků maximálně dvaceti buněk, které náhodně rozmístí do mřížky.
Ovšem nejdůležitější metodou je \emph{get\_future\_grid()}, která ze současného stavu automatu vypočítá ten následující. V ní je definováno chování modelu i jeho změna při vykonávání jednotlivých experimentů.

\subsection{Třída Image}
Tato třída zajišťuje zobrazení grafického výsledku jednotlivých běhů simulace. Její hlavní metodou je \emph{create\_image()}, která vykresluje současný stav mřížky automatu.


\section{Podstata simulačních experimentů a jejich průbeh}
\subsection{Postup experimentování}
\subsection{Dokumentace jednotlivých experimentů}
todo idk
\subsubsection{Experiment XXX}
\subsubsection{Experiment YYY}
\subsubsection{Experiment ZZZ}
\subsection{Zhodnocení experimentů}

\section{Závěr}


\section{Literatura}


\newpage
\bibliographystyle{czplain}
\renewcommand{\refname}{\section{Literatura}}
\bibliography{dokumentace}
\end{document}