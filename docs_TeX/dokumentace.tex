\documentclass[a4paper,11pt]{article}

\usepackage[left= 1.5cm,text={18cm, 25cm},top=2.5cm]{geometry}
\usepackage[utf8]{inputenc}
\usepackage[T1]{fontenc}
\usepackage{times}
\usepackage{paralist}
\usepackage{graphicx}
\usepackage{textcomp}
\usepackage{enumitem}
\usepackage{amssymb}
\usepackage{amsmath}
\usepackage{xcolor}
\usepackage[ddmmyyyy]{datetime}
\usepackage{array}
\pagestyle{plain}
\pagenumbering{arabic}
\usepackage[czech]{babel}
\usepackage{lmodern}
\usepackage{float}


\renewcommand*\contentsname{Obsah}

\newdateformat{mydate}{\twodigit{\THEDAY}.{ }\shortmonthname[\THEMONTH] \THEYEAR}

\pagenumbering{arabic}

\usepackage{url}
\DeclareUrlCommand\url{\def\UrlLeft{<}\def\UrlRight{>} \urlstyle{tt}}

% \usepackage{indentfirst}

\begin{document}
\selectlanguage{czech}

\begin{titlepage}
\begin{center}
    {\Huge \textsc{Vysoké učení technické v Brně}}
\vspace{\stretch{0.01}}
    
    {\LARGE \uppercase{FAKULTA INFORMAČNÍCH TECHNOLOGIÍ}}
    
\begin{figure}[h]
\vspace{5.0cm}
\centering
\includegraphics[scale=0.15]{logo.png}
\vspace{-10.0cm}
\end{figure}
    
\vspace{\stretch{0.382}}
	{\LARGE Projekt IMS, 2019Z}
\vspace{\stretch{0.02}}

	{\Huge \textbf{10 - Celulární automaty}}
\vspace{\stretch{0.02}}\\

{\LARGE {Hrabošová krize}}\\

\begin{figure}[h]
\centering
{\Large {\mydate\today}}
\vspace{6cm}
\end{figure}

\end{center}
\begin{compactitem}
\item[] \textbf{Autoři:}
\item[] Marek Petr, xmarek66
\item[] Vanický Jozef, xvanic09
\end{compactitem}

\end{titlepage}

\tableofcontents
\newpage

\section{Úvod}
Táto práca vznikla v rámci predmetu Modelování a simulace na Fakulte informačných technológií VUT v Brne. Práca popisuje model (\cite{slajdy}, snímok č. 7) celulárneho automatu (\cite{slajdy} č. 209), ktorého úlohou je predikcia intenzity populácie hraboša poľného \textit{(Microtus arvalis)} a jeho regulácia. Cieľom práce je porovnanie výsledkov experimentov - účinnosti jednotlivých protiopatrení voči hrabošovi poľnému na modeli popísanom v článku \verb|\cite{}|. Ide o protiopatrenia hĺbkovej orby, plytkej orby a použitie chemickej látky - Stutox II s účinnou látkou Fosfidom zinečnatým. Tieto protiopatrenia sú ďalej popísané v článkoch \verb|\cite{}| \verb|\cite{}|. 
\subsection{Zdroje faktů}

\subsection{Ověření validity/funkčnosti}


\section{Rozbor tématu, použitých metod a technologií}
\subsection{Popis použitých postupů}

\subsection{Popis použitých metod a technologií}


\section{Koncepce modelu}
todo

\section{Architektura simulačního modelu}
Simulátor je složen ze dvou tříd \emph{Grid} a \emph{Cell}. Třída \emph{Cell} implementuje jednu buňku celulárního automatu. Obsahuje informace o stavu této buňky, tedy její souřadnice a hustotu. Buňka odpovídá 1x1m pole. Třída grid reprezentuje mřížku celulárního automatu a zodpovídá za jeho chování. Také obsahuje základní informace modelu, například velikost mřížky, porodnost, úmrtnost a další. Dále se v ní nachází metody umožňující spustit a řídit simulaci. Nejdůležitější z nich je \emph{get\_future\_grid}, která ze současného stavu automatu vypočítá ten následující.
\subsection{Třída Grid}
\subsection{Třída Cell}
todo


\section{Podstata simulačních experimentů a jejich průbeh}
\subsection{Postup experimentování}
\subsection{Dokumentace jednotlivých experimentů}
todo idk
\subsubsection{Experiment XXX}
\subsubsection{Experiment YYY}
\subsubsection{Experiment ZZZ}
\subsection{Zhodnocení experimentů}

\section{Závěr}


\section{Literatura}


\newpage
\bibliographystyle{czplain}
\renewcommand{\refname}{\section{Literatura}}
\bibliography{dokumentace}
\end{document}